\documentclass{scrartcl}
\usepackage[table,xcdraw]{xcolor}
\usepackage{amsmath}
\usepackage{amssymb}
\usepackage{unicode-math}
\setmathfont{XITS Math}
\usepackage{tikz}
\usetikzlibrary{positioning}
\usepackage{listings}

\newcommand{\selection}{\sigma}
\newcommand{\projection}{\pi}
\newcommand{\rename}{\rho}
\newcommand{\join}{\Join}
% \fullouterjoin defined by unicode-math
\newcommand{\semijoin}{\ltimes}
\newcommand{\groupby}{\Gamma}

\newcommand{\mtt}[1]{\text{\texttt{#1}}}

\setlength{\parindent}{0pt}

\begin{document}

\section*{Exercise 1}

The final results are marked in the Table~\ref{exercise1}. The best join tree according to
$C_{out}$ is $(A \join B) \join C$.

\begin{table}[]
\centering
\caption{DPsub Table}
\label{exercise1}
\begin{tabular}{@{}llll@{}}
\rowcolor[HTML]{F8A102}
Set & Plan & Cost & Card \\
A & $A$ & 0 & 10 \\
B & $B$ & 0 & 20 \\
A,B & $A \join B$ & 100 & 100 \\
C & $C$ & 0 & 100 \\
A, C & $A \join C$ & 1000 & 1000 \\
B, C & $B \join C$ & 200 & 200 \\
\rowcolor[HTML]{FFCC67}
A, B, C & $(A \join B) \join C$ & 1100 & 1000 \\
\rowcolor[HTML]{FFCC67}
 & $(A \join C) \join B$ & 2000 & 1000 \\
\rowcolor[HTML]{FFCC67}
 & $(B \join C) \join A$ & 1200 & 1000 \\
\end{tabular}
\end{table}

\section*{Exercise 3}

Running our program on the given query gives the following result:
\begin{lstlisting}
J0 = orders o |><| customer c  Cost: 150000
Result = lineitem l |><| J0  Cost: 750572
\end{lstlisting}

\end{document}
